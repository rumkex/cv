% FortySecondsCV LaTeX template
% Copyright © 2019-2020 René Wirnata <rene.wirnata@pandascience.net>
% Licensed under the 3-Clause BSD License. See LICENSE file for details.
%
% Please visit https://github.com/PandaScience/FortySecondsCV for the most
% recent version! For bugs or feature requests, please open a new issue on
% github.
%
% Contributors
% ------------
% * ifokkema
% * Bertbk
% * Hespe
%
% Attributions
% ------------
% * fortysecondscv is based on the twentysecondcv class by Carmine Spagnuolo
%   (cspagnuolo@unisa.it), released under the MIT license and available under
%   https://github.com/spagnuolocarmine/TwentySecondsCurriculumVitae-LaTex
% * further attributions are indicated immediately before corresponding code


%-------------------------------------------------------------------------------
%                             ADDITIONAL PACKAGES
%-------------------------------------------------------------------------------
\documentclass[
	a4paper,
	% showframes,
	% vline=2.2em,
	% maincolor=cvgreen,
	% sidecolor=gray!50,
	% sectioncolor=red,
	% subsectioncolor=orange,
	% itemtextcolor=black!80,
	% sidebarwidth=0.4\paperwidth,
	% topbottommargin=0.03\paperheight,
	% leftrightmargin=20pt,
	% profilepicsize=4.5cm,
	% profilepicborderwidth=3.5pt,
	% profilepicstyle=profilecircle,
	% profilepiczoom=1.0,
	% profilepicxshift=0mm,
	% profilepicyshift=0mm,
	% profilepicrounding=1.0cm,
]{fortysecondscv}

% improve word spacing and hyphenation
\usepackage{microtype}
\usepackage{ragged2e}

% uncomment in case you don't want any hyphenation
% \usepackage[none]{hyphenat}

% take care of proper font encoding
\ifxetexorluatex
	\usepackage{fontspec}
    \usepackage{polyglossia}
    \setmainlanguage{russian} 
    \setotherlanguage{english}
	\defaultfontfeatures{Ligatures=TeX}
    
    \setmainfont{Linux Libertine O}
    \setromanfont{Linux Libertine O} 
	\newfontfamily{\cyrillicfont}{Linux Libertine O} 
	\newfontfamily{\boldfont}{Linux Libertine O} 
\else
	\usepackage[utf8]{inputenc}
	\usepackage[T1]{fontenc}
%	\usepackage[sfdefault]{noto} % use noto google font
\fi

% enable mathematical syntax for some symbols like \varnothing
\usepackage{amssymb}

% bubble diagram configuration
\usepackage{smartdiagram}
\smartdiagramset{
	% default font size is \large, so adjust to harmonize with sidebar layout
	bubble center node font = \footnotesize,
	bubble node font = \footnotesize,
	% default: 4cm/2.5cm; make minimum diameter relative to sidebar size
	bubble center node size = 0.4\sidebartextwidth,
	bubble node size = 0.25\sidebartextwidth,
	distance center/other bubbles = 1.5em,
	% set center bubble color
	bubble center node color = maincolor!70,
	% define the list of colors usable in the diagram
	set color list = {maincolor!10, maincolor!40,
	maincolor!20, maincolor!60, maincolor!35},
	% sets the opacity at which the bubbles are shown
	bubble fill opacity = 0.8,
}


%-------------------------------------------------------------------------------
%                            PERSONAL INFORMATION
%-------------------------------------------------------------------------------
%% mandatory information
% your name
\cvname{Никита Терешин}

% job title/career
\cvjobtitle{Разработчик ПО}

%% optional information
% profile picture
% \cvprofilepic{pics/profile.png}

% NOTE: ordering in sidebar will mimic the following order
% date of birth
\cvbirthday{9 декабря 1992}
% short address/location, use \newline if more than 1 line is required
\cvaddress{Москва, м. Раменки}
% phone number
\cvphone{+7 (903) 509 30 42}
% email address
\cvmail{nikita.tereshin@gmail.com}

%-------------------------------------------------------------------------------
%                              SIDEBAR 1st PAGE
%-------------------------------------------------------------------------------
% add more profile sections to sidebar on first page
\addtofrontsidebar{
	% include gosquare national flags from https://github.com/gosquared/flags;
	% naming according to ISO 3166-1 alpha-2 country codes
	\graphicspath{{pics/flags/}}

	\profilesection{О себе}
    \aboutme{
    Питонист с талантом к обучению. Имею значительный опыт разработки на C\#, и при необходимости не гнушаюсь C/C++. Ищу вакансию Python-разработчика, на которой потребуется решать интересные и новые задачи.
    }

	% social network accounts incl. proper hyperlinks
	\profilesection{Контакты}
		\begin{icontable}{2.5em}{1em}
			\social{\faTelegram}
				{https://t.me/rumkex}
				{@rumkex}
			\social{\faSkype}
			    {skype:iincognito?chat}
			    {iincognito}
			\social{\faGithub}
				{https://github.com/rumkex/}
				{rumkex}
		\end{icontable}

	\profilesection{Языки}
	\pointskill{\flag{flat/RU.png}}{Русский}{5}
	\pointskill{\flag{flat/GB.png}}{Английский}{5}

	\profilesection{Hard skills}
		\pointskill{\faPython}{Python}{5}
		\skill[1.5em]{\faCompress}{Web-разработка}
		\skill[1.5em]{\faCompress}{Обработка данных}
		\pointskill{\faCube}{C\#}{5}
		\pointskill{\faCube}{C++}{3}
		\pointskill{\faCube}{HTML/CSS/JS}{3}
		\skill[1.5em]{\faCompress}{React}
		\skill{\faDatabase}{SQL}
		\skill{\faSync}{DevOps}
		\skill{\faSitemap}{Администрирование (Linux)}

	\profilesection{Soft skills}
		\pointskill{\faChild}{Легкообучаемость}{3}[4]
		\pointskill{\faCompress}{Координация \\ \hspace*{0.65cm} небольших команд}{2}[4]
}


\begin{document}

\makefrontsidebar

\cvsection{Опыт работы}
\begin{cvtable}[3]
	\cvitem{2017 --\\ наст. время}{ООО "Платформа Качества"}{Разработчик}{
	    Создаю ПО под заказ для различных клиентов и участвую в разработке собственных проектов компании, включая примеры ниже. В последних проектах являюсь руководителем небольшой команды из двух других разработчиков. Среди других моих обязаннностей - настройка и поддержка внутренней инфраструктуры компании (VPN, self-hosted сервисы для CI, контейнеризация и т.п.).
	}
	\cvitem{2018 --\\ наст. время}{МГУ им. Ломоносова}{Младший научный сотрудник}{
	    Обрабатываю кучу GNSS-данных. Разрабатываю ПО для анализа данных.
	    \\
    	\textit{Основные технологии: Numpy/Scipy/Matplotlib}
	}
\end{cvtable}

\cvsection{Примеры проектов}

\textbf{Проект:} веб-сервис, предоставляющий пользовательский интерфейс, систему событий и очереди заданий, и диагностические функции для IoT-решения в офисном пространстве (текущий проект).

\textit{Мой вклад}: архитектура API и БД, реализация основных описанных выше систем, помощь junior-разработчику и менторство над ним.

\textit{Основные технологии: FastAPI, SQLAlchemy, Celery, RabbitMQ}
\\[0.5em]
\textbf{Проект:} сервис сбора данных датчиков климата внутри помещений, предоставляющий дэшборд с различными срезами аналитики по этим данным.

\textit{Мой вклад}: бэкенд, частичная разработка фронтенда, руководство командой из двух разработчиков.

\textit{Основные результаты}: развертка проекта на двух тестовых локациях c нагрузкой порядка 100000 измерений в день, несколько успешных демонстраций потенциальным клиентам.

\textit{Основные технологии: Flask, SQLAlchemy}
\\[0.5em]
\textbf{Проект:} внутренний веб-сервис для конфигурируемой обработки передаваемых заказчику табличных данных для загрузки в единую БД. Сервис интегрируется с  \textenglish{ActiveDirectory} и несколькими \textenglish{HTTP API} сторонних сервисов.

\textit{Мой вклад}: полная реализация бэкенда, тесты, развертка, интеграции, техподдержка.

\textit{Основные результаты}: разработанный сервис успешно решил несколько существовавших проблем с производительностью и юзабилити, которые описали сотрудники заказчика, и заменил существующий инструмент.

\textit{Основные технологии: C\#, ASP.NET Core, EF Core, LDAP}
\\[0.5em]
\textbf{Проект:} CLI-инструменты для управления и взаимодействия с BLE меш-сетью.

\textit{Мой вклад}: разработка полного решения

\textit{Основные результаты}: разработчики и тестировщики компании-клиента активно используют разработанный инструментарий, говоря, что он значительно более наполнен функционально и удобен в использовании. Он также используется в боевой среде для задач техподдержки.

\textit{Основные технологии: AsyncIO, Python C API}

\cvsection{Образование}
\begin{cvtable}[1.5]
	\cvitem{2015 -- 2018}{Аспирантура: физика}{МГУ им. Ломоносова}
		{Специализация - физика атмосферы. Дипломная работа: построение 2D-карт атмосферного влагосодержания по данным GPS-приемников.}
	\cvitem{2009 -- 2015}{Специалитет: физика}{МГУ им. Ломоносова}{ }
\end{cvtable}


\cvsection{Хобби}
\begin{cvtable}
	\cvitemshort{Байдарки}{Идеально для летних выходных}
	\cvitemshort{Горные лыжи}{Когда гравитация работает на тебя}
	\cvitemshort{Игры}{Компьютерные, настольные... главное - не бегать}
\end{cvtable}


% \newpage
% \makebacksidebar

\end{document}
